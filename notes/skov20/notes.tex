\documentclass{article}
\begin{document}

Abstr:

``we assign 84.5\% of fragments to an Altai or Vindija Neanderthal
origin and 3.3\% to Denisovan origin; 12.2\% of fragments are of
unknown origin. We find that Icelanders have more Denisovan-like
fragments than expected through incomplete lineage sorting. This is
best explained by Denisovan gene flow, either into ancestors of the
introgressing Neanderthals or directly into humans."

``although the overall rate of mutation was similar in humans and
Neanderthals during the 500 thousand years that their lineages were
separate, there were differences in the relative frequencies of
mutation types---perhaps due to different generation intervals for
males and females.''

1: ``We used a two-state hidden Markov model that is not conditioned
on DAV genomes to search for archaic fragments independently in 55,132
haploid genomes from 27,566 sequenced Icelanders that were sequenced
to 30X and accurately phased using long-range phasing.''

2: ``The 14,422,595 candidate archaic fragments correspond to 112,709
unique fragments (based on start-end positions), have a combined
length of 1,818 Gb and cover 1.179 Gb (48.2\%) of the callable
genome.''

2: ``Callable positions in the genome were, on average, covered by 743
archaic fragments, corresponding to an average frequency of 1.34\% in
the 55,132 haploid genomes analysed. These are probably
underestimates, because our method misses archaic fragments that are
short or not sufficiently divergent from non-archaic
fragments. Simulations suggest a false-negative rate of 28.5\%,
suggesting that the frequency of archaic fragments may be closer to
1.9\%.''

2--3: ``The average nucleotide diversity between overlapping archaic
fragments, when considering DAV and DAV-linked variants is $2.00
\times 10^{-5}$ which is similar to the heterozygosity in Neanderthals
(Altai Neanderthal, $1.63 \times 10^{-5}$; Vindija Neanderthal, $1.71
\times 10^{-5}$), and the Altai Neanderthal, $1.89\times 10^{-5}$).
Some genomic regions contained highly diverse archaic fragments
indicative of multiple introgressing haplotypes.''

3: ``we find genomic regions with up to six different archaic
haplotypes---indicating that multiple archaic individuals were
involved in the introgression.''

This must be an underestimate. If archaic DNA had been neutral, then
\[
6/N = 0.02 \qquad\hbox{or}\qquad N = 300
\]
would be an estimate of the size of the modern population that
received archaic gene flow. But archaic DNA seems to have been
selected against, so the initial fraction must have been larger than
0.02. This would imply an even smaller value of $N$.

3: ``Because our method of detection is not dependent on existing
archaic genomes, we are able to detect mosaic fragments that switch
similarity with different DAV genomes along their length We find that
18.9\% of fragments are mosaic.''

This is remarkable. Consistent with other evidence of interbreeding
between Neanderthals and Denisovans.

To exclude incomplete lineage sorting, they did simulations and
concluded:

``The observed characteristics of Denisovan-like fragments in
Icelanders are not compatible with a simple introgression from a
Vindija-like group without that population having had prior admixture
with a Denisovan-like group (Supplementary Information 3.3.3 and
Supplementary Fig. 3.1.1). An equally intriguing scenario that cannot
be ruled out is direct admixture from a Denisovan-like group into the
common ancestors of non-Africans.''

Under the latter hypothesis, the mosaic haplotypes would have been
formed by recombination within the modern population. This seems less
likely to me, because only a small fraction of the modern genome is
archaic in origin.

3: ``We estimated the effective population sizes ($N_e$) of different
archaic groups using pairwise differences between the archaic
fragments identified in Icelanders and the high-coverage DAV genomes
(Supplementary Information 2.7) and divergence times estimated in
previous studies1,18 (Supplementary Information 2.7.2). We find that
Neanderthals had a relatively small $N_e$ of 2,000--3,000 individuals,
in agreement with previous pairwise sequential Markov chain analyses
of Neanderthal genomes.''

3: ``Our greater sample size ($n = 27,566$ compared with $n = 502$ in
the 1000 Genomes Project) enables a more fine-scale identification of
genomic regions with very little or no archaic introgression (archaic
deserts). Searching for 1-Mb windows with no fragments containing DAV
variants, we found 282 distinct archaic deserts covering 570~Mb
(23.3\% of the callable genome) (Extended Data Fig. 5 and
Supplementary Data 4). The X chromosome is particularly devoid of
archaic introgression as previously reported.''

Not sure I understand the following. What do the authors mean by
``spanned?''

3: ``the proportion of archaic fragments spanned by DAV-linked and DAV
variants was higher in regions with a greater recombination rate
(Spearman’s $\rho = 0.15$, $P = 4.4 \times 10^{-54}$) and the
nucleotide diversity of archaic fragments was also higher in these
regions (Spearman’s $\rho = 0.18$, $P = 1.3 \times 10^{-79}$)
(Extended Data Fig. 6). Taken together, these observations indicate
that non-deleterious archaic variants were more likely to be retained
in the human gene pool when they could be uncoupled from deleterious
archaic variants by recombination.''

4: ``The most-extreme ratios are observed for C$>$G and CpG$>$T
variants, which have the lowest (1:1) and highest (6:1)
paternal-to-maternal age effect ratios for de novo mutations in
contemporary humans (the mean ratio across all types is 3:1).  These
differences are consistent with mothers having been older and fathers
younger in the Neanderthal lineage than in the human lineage, although
other causes cannot be ruled out.''

4: ``It has been suggested29-31 that Neanderthals accumulated more
deleterious mutations than humans because of less efficient selection
due to a smaller $N_e$. We did not find an excess of deleterious
variants in the archaic fragments identified in Icelanders ($\chi^2$
test, $P=0.71$). \ldots Therefore, if introgressing Neanderthal
fragments had more deleterious variants at the time of introgression,
these have already been removed by purifying selection.''

4: ``In the second step of filtering, we tested whether any correlated
($r^2 > 0.2$) non-archaic variant within a 2-Mb radius better
accounted for an association than the strongest archaic candidate
variant. For 550 association signals, we found a non-archaic variant
in high linkage disequilibrium with the strongest archaic
candidate. Of these, 431 involved a non-archaic variant with a
substantially stronger association ($>$10-fold) with the phenotype in
question. For 33 cases, the non-archaic variant had a stronger
association ($\leq$10-fold) with the phenotype. In another 86
instances, archaic variants were disregarded because a highly
correlated ($r^2 > 0.9$) non-archaic variant with a slightly stronger
association was identified. This left 101 archaic variants.''

There were several steps in this filtering process.

5: ``Finally, we pruned variants that were not in high linkage
disequilibrium with other archaic variants ($r^2 > 0.9$), leaving only
five independent archaic variants likely to be a true source of
phenotypic association.''

6: ``We could therefore only validate the archaic origin of 3 out of
26 previously reported association findings attributed to archaic
variants (Supplementary Data 8). This highlights the importance of
considering flanking non-archaic variants when assigning an archaic
origin to a phenotype association.''

6: ``To assess the genome-wide effect of archaic introgression on
phenotype variation, we also counted the number of archaic-derived
alleles carried by each contemporary Icelander and tested this
polygenic score of archaic ancestry for association with each of the
271 phenotypes. In the case of height, for example, this test would
reveal whether the surviving fragments from archaic introgression
resulted in taller or shorter contemporary Icelanders. After adjusting
for the number of tests, we found no evidence for association between
the polygenic score and any of the 271 tested phenotypes.''

6: ``However, the considerable proportion of archaic fragments that
are closer to the Denisovan genome cannot be explained by incomplete
lineage sorting (Supplementary Information 3.3.3.1). Rather, they
require Denisovan introgression, either directly into humans or into
Neanderthals who later mixed with humans, which must have occurred
soon after they migrated out of Africa, because its signal is found in
all contemporary non-African populations from the Simons Genome
Diversity Project.''

6: ``The similar mutation rates in Neanderthals and modern humans
imply that the apparent slowdown in the human mutation rate36 is
unlikely to have occurred between 500,000 and 55,000 years
ago. However, differences in the mutation spectrum of archaic and
non-archaic fragments raise the possibility of long-term differences
in male and female generation intervals between the species.''

6: ``Finally, our most far-reaching conclusion is that archaic
introgression has a relatively minor effect on phenotypic variation in
contemporary humans. Given the non-random genomic distribution of
archaic fragments in contemporary Icelanders, it follows that this
influence must have been greater in the past.''

Archaic haplotypes haplotypes are detectable only if they are long,
which suggests that they ought to be more common in genomic regions
with low rates of recombination. But if I understand them correctly,
the opposite is true, presumably because selection has removed archaic
haplotypes more effectively in regions with low rates of
recombination. This suggests that many short archaic haplotypes have
been missed in regions of high recombination. These may affect
function.

S37: To estimate the effective sizes of archaic populations, Skov et
al compare the introgressed segments to high-coverage archaic
sequences. Their inference builds on
\[
\hbox{number of mutations} = \mu L T
\]
where $\mu$ is the
mutation rate, $L$ is the length of the introgressed fragment, and $T$
is the time since the most recent common ancestor of moderns and the
ancient genome. 

This doesn't seem quite right. If the modern and archaic individual
had lived at the same time, it should be
\[
\hbox{number of mutations} = \mu L \cdot 2T
\]
because mutations can occur either on the branch leading to moderns or
on that leading to the archaic.  But this is still wrong, because
mutations stopped accumulating on the archaic branch when the archaic
individual died 50--100~ky ago. I would have written
\[
\hbox{number of mutations} = \mu L (2T - t)
\]
where $t$ is the age of the archaic genome. The second correction is
minor, so their estimate of $T$ is roughly 2$\times$ too large, and
this leads to an overestimate of $N_e$.

\end{document}
S41: ``However, N1, N2 and N3 appear to be overestimated. There are several
factors which contribute to this. Firstly, we are only keeping SNP
dense fragments (removing all fragments with a mean posterior
probability less than 0.9). This means that we more likely call SNP
dense fragments in the first place and these will have a higher
TMRCA.'' 
