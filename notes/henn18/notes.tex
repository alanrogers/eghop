\documentclass{article}
\begin{document}
Clarifying distinct models of modern human origins in Africa

Brenna M Henn, Teresa E Steele, and Timothy D Weaver

``Here, we describe four possible models for the origins of Homo
sapiens in Africa based on published literature from paleoanthropology
and human genetics. We briefly outline expectations for data patterns
under each model, with a special focus on genetic data.''

Model 1: African multiregionalism

Model 2: Single origin. Range expansion with local extinctions.

Model 3: Single origin, range expansion with regional persistence.

Model 4: Archaic hominin admixture in Africa.

\section{Model 1: African multiregionalism}

``Fossils exhibiting at least some diagnostic modern traits are found
in multiple regions in Africa []. The presence of modern traits in
multiple regions is to be expected if multiple localities were
involved in modern human origins''

``Under an African Multiregional model (Figure 1a), we would expect
deep population divergence estimates between groups found in different
African geographic regions, accompanied by relatively high migration
rates between proximate regions.''

``the only paper to test the fragmentation versus single origin
hypothesis found that the pan-African origin model had significantly
lower support than a single origin model [21].''

``Furthermore, under an African Multiregional model, estimates of the
ancestral effective population size of humans would be extremely
large. \ldots Estimates of the ancestral human population size remain
relatively small (typically a range of 9000--30\,000)
[20,21,42,43]. As discussed by Sjodin et al. [21], using an assumption
that Ne reflects 10\% of the census size, the maximum estimate of
32,500 would result in a population density of 1.4 individuals per
100~km$^2$ which is lower even than population densities estimated for
African hunter-gatherers who have among the lowest population
densities today (e.g. Dobe !Kung ${\sim}6.6 \times 100$~km$^2$, Hadza
${\sim}30 \times 100$~km$^2$ [44]). Densities could be increased by
assuming that only a fraction of the African continent is habitable;
however, this then suggests that strong ecological/physical barriers
exist. Hence, a model of pan-African prehistoric migration is
unlikely.''

Or maybe habitable regions connected by corridors of migration, such
as the Nile Valley or the Silk Road.

\section{Model 2: Single origin. Range expansion with local
  extinctions}

``Model 2 is supported by evidence that the ancestral human population
has a relatively small Ne, estimates of population divergence
systematically find that the southern African KhoeSan are the most
divergent human population, and estimates of the time of population
divergence remain relatively young (Table 1).''

Young? Schlebusch et al got 300kya.


\section{Model 3: Single origin, range expansion with regional
  persistence}

``Under Model 3, we would expect to see clinal patterns of genetic
diversity and coalescence radiating from a singular region of
Africa. However, the phylogeographic pat- terns will contain
topologies that reflect deep coalescent events in regions outside of
the immediate source. The fraction of these ‘inconsistent’ topologies
will reflect the amount of gene flow from the previously isolated
human populations into the expanding source.''

``One example that might support Model 3 is the recent observation of
$10\% divergent ancestry in western Africans which does not fit a
monophyletic branching model of population divergence from a southern
African source (represented by the ancestors of the KhoeSan)
[40,52].''

``Additionally, the highly divergent A00 Y-chromosome is basal to the
remainder of human Y-chromosomes and has an estimated tMRCA of 250 000
years ago. Its distribution is circumscribed to Cameroon (and
African-American descendants)---and the lack of diversity on the
background of A00 suggests an extremely small population or very low
gene flow into the modern human population''

Or recent introgression.

\section{Model 4: Archaic hominin admixture in Africa}

``However, it is not clear what barriers (geographic, behavioral)
allowed modern and archaic evolutionary lineages to remain distinct
for hundreds of thousands of years before there was recent
admixture.''

``Hsieh et al. consider the top 1\% of S$^{*}$ outliers in central
African Pygmy genomes [55]; they find evidence for a single pulse of
gene flow into central Africans $\sim$9000 years ago. These genetic
regions tend to have very old coalescent ages, approximately 1 million
years old. Rather than supporting evidence for extensive gene flow
(i.e. Model 1), these data rather would support strong population
structure for tens of thousands of years between hominin
species---followed by a one pulse or a two-pulse admixture event with
low migration rates.''

\end{document}
