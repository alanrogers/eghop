\documentclass{article}
\begin{document}

``Human gut microbiome composition and functional potential are more
similar to those of cercopithecines, a subfamily of Old World monkey,
particularly baboons, than to those of African apes.'' The microbiomes
of humans are also more variable than those of other primates.

``a recent systematic examination of data from 18 primate species
across the phylogeny reveals that less than 3\% of microbial taxa
defined by 97\% sequence similarity co-diversify with hosts."

So most microbial taxa are generalists. This is interesting, because
many other parasites are specialists and co-speciate with their
hosts. Eg.~tapeworms. Why would microbial parasites be less
specialized?

``To the extent that hosts of the same phylogenetic group share
physiological dietary adaptations, they will also share gut microbial
traits.''

``Chimpanzees and bonobos, in particular, are described as ripe-fruit
specialists, consuming high percentages of fruit even when
availability is reduced [26]. These differences in feeding ecology are
associated with differences in digestive physiology. For example,
salivary amylase expression in chimpanzees is one third of that in
humans [27], and both chimpanzees and bonobos have rapid intestinal
transit time relative to body mass, which has been associated with
their highly frugivorous diet''

``humans occupy an ecological niche more similar to that of distantly
related cercopithecines (a subfamily of Old World monkey)
[30,31,32]. Cercopithecines inhabit grasslands with varying degrees of
woody cover and utilize an omnivorous diet that includes underground
plant storage organs of C4 grasses and sedges [33]. It has been
previously argued that a subset of cercopithecines, the papionin
primates (geladas---Theropithecus gelada and baboons---Papio spp.),
are the best ecological analogues for hominin ancestors.''

``Gut microbiomes of industrialized populations clustered away from
all other primates while gut microbiomes of non-industrialized
populations clustered with apes and Old World monkeys (Additional file
2: Figures S1, S2). Given that industrialized humans were clearly
outliers and that New World monkeys and lemurs had limited
similarities to humans, we removed these samples from all further
analyses.''

``Repeating the analysis with only non-industrialized human
populations, Old World monkeys, and apes demonstrated that the
taxonomic composition of the human gut microbiome (16S rRNA gene
amplicon data) was more similar to that of cercopithecines than apes
(Fig. 1, Additional file 2: Figure S8). Although the gut microbiome of
cercopithecines exhibited higher taxonomic diversity than that of both
humans and apes.''

Fig 1: You need a tree to construct unifrac distances. But then you
use these distances to construct another tree. Why? How did you build
the original tree?

Why an ultrametric tree? Is it fair to assume that these taxa evolve
at the same rate?


\end{document}
