\documentclass{article}
\begin{document}
100,000 years of gene flow between Neandertals and Denisovans in the
Altai mountains

Benjamin Peter

``While the two earliest Denisovans both have substantial and recent
Neandertal ancestry, I find signatures of admixture in all archaic
genomes from the Altai, demonstrating that gene flow also occurred
from Denisovans into Neandertals. This suggests that a contact zone
between Neandertals and Denisovan populations persisted in the Altai
region throughout much of the Middle Paleolithic. In contrast, Western
Eurasian Neandertals have little to no Denisovan ancestry. As I find
no evidence of natural selection against gene flow\ldots''

I don't understand Fig. 1c. Also, in 1a, does ``Cont'' refer to
``contamination?'' And are these values estimated internally or taken
from previous publications.

Ancient admixture tracts are harder to identify than recent ones,
because they are shorter. I imagine this is problem is more severe
with low-quality genomes. What is the time scale over which your
method is feasible?

Ben says that his method works well for tracts as small as 0.1cM. This
corresponds to a recombination rate of about r=0.001. At this scale,
half the LD is gone in (log 2)/r = 700 generations. By 3 half-lives
(2000 generations) after the admixture event, there wouldn't be many
tracts as long as 0.1cM, and the method should lose sensitivity. For
contaminated DNA, you need tracts of length 0.2cM, and you can only
see back about 1000 generations. Is that about right?

However, all the tracts found in Chakgryskaya were smaller than 0.1cM,
so the method is apparently useful at longer time scales.

``The average length of Neandertal ancestry tracts in Denisova 2
suggests that most Neandertal ancestry dates to around 1,500 years
prior to when Denisova 2 lived.''

Because longer tracts are easier to find, the average length of
discovered tracts should be longer than the unbiased average. Does
this impart a downward bias to estimates of the timing of admixture?

``A lower total of 3.8 cM (4.8Mb) of Denisovan introgressed material
is found in Chagyrskaya 8, a more recent Neandertal from the Altai
mountains33. The inferred tracts are small, with the longest tract
measuring 0.83 cM (Extended Data Fig. 7d), suggesting that this gene
flow happened several tens of thousands of years before Chagyrskaya 8
lived. In contrast, little to no Denisovan ancestry is detected in
eight Western Eurasian Neandertal genomes4,20,29 dating from between
40,000 and 120,000 years ago.''

``In three of these genomes (Goyet Q56-1, Spy 1 and Les Cottes), the
centromere of chromosome 10, a region implicated in gene flow between
archaic and modern humans34, is identified as introgressed from
Denisovans.''

I'm surprised they even \emph{have} sequence from the centromeres.

``As there is also no evidence for reduced introgressed ancestry on
the X-chromosome (p=0.47, permutation test), no association of
introgressed regions with levels of background selection35 (p>0.14,
permutation test) and no evidence that introgressed tracts correlate
with any functional annotation category (GO-enrichment analysis36,
hypergeometric test, p>0.05 for all categories), and no significant
association of introgression tracts with regions depleted for
Neandertal ancestry in modern humans (p=0.317, permutation test),
there is no evidence of negative fitness consequences of
Neandertal-Denisova matings.''



\end{document}
