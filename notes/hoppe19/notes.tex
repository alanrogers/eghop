\documentclass{article}
\begin{document}
\begin{center}
Notes on: 

Hoppe et al. 2019. Topic choice contributes to the lower rate of NIH
awards to African-American/black scientists.
\end{center}

``There remains a lower rate of funding of National Institutes of
Health R01 applications submitted by African-American/black (AA/B)
scientists relative to white scientists. \ldots Disparate outcomes
arise at three of the six [stages of the application process]:
decision to discuss, impact score assignment, and a previously
unstudied stage, topic choice. Notably, AA/B applicants tend to
propose research on topics with lower award rates. \ldots Topic choice
alone accounts for over 20\% of the funding gap.''

``At the discussion stage, applications from AA/B scientists receive
poorer overall impact scores on average than those of WH scientists
(38.4 � 13.4 SD and 35.2 � 12.6 SD, respectively, P < 0.0001; table
S4). Cumulatively, the lower submission rates, lower average
discussion rates, and lower impact scores result in applications from
AA/B scientists receiving R01 funding at approximately half the rate
(0.5-fold) of those from WH scientists (Fig. 1).''

``Notably, 37.5\% of all applications from AA/B scientists mapped to
only 8 of the 150 topic clusters (compared to a random distribution,
$P < 0.0001$). Of those eight clusters, six had award rates that were
significantly below the NIH average (table S6). There was therefore a
trend among AA/B applicants to submit applications on topics that
experience lower funding rates, irrespective of the study section to
which they were assigned.''

``Defining words in the eight clusters with the highest percentage of
applications from AA/B applicants include socioeconomic, health care,
disparity, lifestyle, psychosocial, adolescent, and risk; these
clusters had funding levels ranging from 11.2 to 17.2\% (table S7). In
contrast, frequently used words in the eight clusters without any AA/B
applicants (see Fig. 3A) include osteoarthritis, cartilage, prion,
corneal, skin, iron, and neuron; these clusters had funding levels
ranging from 12.5 to 28.7\%.''

``Numerous analyses have shown that study section-assigned scores do
not discriminate between grants that go on to produce work of higher
versus lower influence.''

``We found that the lowest-success topics produced papers that were
typically more influential (higher median RCR) than those from the
highest-success topics (fig. S8). Furthermore, for topics in either
the highest or lowest quintile of award rates, plotting the percentile
score of each award against the median RCR of all papers it produced
shows the complete absence of a correlation between study section
assessment and future productivity (fig. S9)''

``The decision point that makes the largest single contribution to the
funding gap is the selection of applications for discussion.''



\end{document}
