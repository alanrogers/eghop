European mitochondria seem to descend mainly from Middle-Eastern
immigrants, who arrived 7000 years ago with the Neolithic. Most
European Y chromosomes, on the other hand, seem to have arrived with
the Indo-European invasion in the Bronze Age. Thus, the Y's and the
mitochondria inhabited separate populations 7000 years ago, and the
autosomes and X's would have been split between these
populations. Would it be possible to extend your new method to unravel
effects such as these?

``Since they have different effective sizes, the X chromosome and
autosomes recover genetic diversity following a population size change
at different rates, even in the absence of sex-bias [17]''

After a population increase, the rate of recovery is governed more by
mutation than by population size. That rate is 2u + 1/2N, which is
approximately 2u if N is large.

The model assumes that entries in the SFS are Poisson, which would
seem to assume that nucleotide sites are independent. Is LD a problem
here?

``In both populations, the best-fitting models are male-biased:
$\tilde p = 0.465$ for Yorbans with an old growth model and $\tilde p
= 0.435$ for Europeans with a bottleneck followed by recent
exponential growth.''

``To reduce the impact of linkage on our inference (specifically, the
differential linkage on the autosomes and the X chromosome), we used a
conventional bootstrap to estimate standard errors of parameters''

Not a moving-blocks bootstrap?

``The best-fitting model for Europeans (CEU) is the complex model (S8
Fig) with a changing sex-bias (Fig 8B): our method infers a
male-biased bottleneck during the Out-of-Africa migration () and a
female bias outside the bottleneck ()'' 
