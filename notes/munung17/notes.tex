\documentclass{article}
\begin{document}
\begin{center}
Notes on: Equity in international health research collaborations in
Africa: Perceptions and expectations of African researchers.

Nchangwi Syntia Munung, Bongani M. Mayosi, Jantina de Vries.
\end{center}

``This study identifies challenges pertaining to equity in research
collaborations which HIC partners should consider when they engage in
international health research in Africa.''

``Between September 2014 and June 2015, 17 face-to-face
semi-structured interviews were conducted with genomics researchers in
Africa. Purposive sampling [26] was used to select research
participants. Research participants were genomics researchers based in
an African research institution and who were directly involved in an
international genomics research and biobanking project.''

``An African researcher said that without international health
research, researchers in Africa, would be sitting all day in their
offices, `reading newspapers' instead of doing research.''

``This has led to situations whereby HIC collaborators had said to
their African collaborators that they were only hired to do the
research and not to be involved in key decision making activities
[38]. This is problematic in that it highlights international health
research as a patronizing and neocolonial activity.''

``about two thirds of our interviewees mentioned that whilst their
collaborators in HICs have the capacity and resources to continue
research on stored samples and the data that were jointly obtained
during the funded collaboration, they would not have access to similar
research funds. The risk, therefore, is that though initial studies
may seek to establish more equitable forms of collaboration, African
researchers may still be marginalized in subsequent research
projects.''

``Unfortunately, in many African research institutions, laboratory
equipment which are primarily obtained through funds from
international health research tend to lie loose with little use once
the projects they were obtained for are over. In the unfortunate case
of a breakdown, equipment worth thousands of dollars are usually
abandoned to the dust for sheer of lack of resources to maintain
them.''



\end{document}
